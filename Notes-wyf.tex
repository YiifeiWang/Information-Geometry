\documentclass[11pt]{article}
\RequirePackage{fancyhdr}
\usepackage{hyperref}
\usepackage{geometry}
\usepackage{amsmath}
\usepackage{amsfonts}
\usepackage{lastpage}
\usepackage{graphicx}
\usepackage{subfigure}
\usepackage{multirow}
\usepackage{tabu}
\usepackage{lipsum}
\usepackage{essay-def}
\usepackage{clrscode}
\usepackage{caption}
\usepackage{appendix}
\usepackage{mathrsfs}
\usepackage{bm}
\usepackage{algorithm}
\usepackage{algorithmic}
\renewcommand{\algorithmicensure}{ \textbf{Output:}} %UseOutput in the format of Algorithm
\renewcommand{\algorithmicrequire}{ \textbf{Input:}} %Use Input in the format of Algorithm
\usepackage{indentfirst}
\RequirePackage[backend=bibtex, style=numeric, maxcitenames=1]{biblatex} 

\pagestyle{plain}

\geometry{
	a4paper, includefoot, hmargin = 2.6cm, top = 2.7cm, bottom = 2cm,
	headheight = 1.5cm, headsep = 0.5cm, footskip = 0.75cm
}
\linespread{1}

\fancypagestyle{plain}{
	\fancyhf{}\renewcommand*{\headrulewidth}{0.75bp}
	\fancyhead[R]{\normalfont{Page \thepage{} of \pageref*{LastPage}}}
	\fancyhead[L]{\normalfont{\cuniversity}}
	\fancyhead[C]{\normalfont{\titlemark}}
}

\bibliography{../SIAM/DeepL}
\usepackage{url}

\newcommand{\cuniversity}{Peking University}
\newcommand{\titlemark}{Notes for Information Geometry}
\author{Yifei Wang 1500010611}
\title{\textbf{\titlemark}}
\date{}

\begin{document}
\maketitle

\section{Manifold, Divergence and Dually flat structure}
\textbf{Question}: what does the inner product mean in (1.93)? We try to provide an answer in the following example.
\begin{example}[Tangent space]
In this example, we will understand $\mfe_i,g_{i,j}(\theta),\mfe^{*i},g^{*i,j}(\theta^*)$ and verify the following relationship (definitions) via a concrete example of probability models. 
$$
\lra{\mfe_i,\mfe^{*j}}=\delta_{i,j}
$$
We assume that $\lra{\cdot,\cdot}$ is the standard Euclidean inner product.
Here we simply take $\psi(\theta)=-\sum_i\log \theta^i$ with $\theta\in\mbR^n_+$. The dual of $\psi$ follows 
$$
\psi^*(\theta^*) = \max_\theta \theta^T\theta^*-\psi(\theta)=-\sum_i(1+\log(-\theta^*_i)).
$$
The relationship between $\theta$ and $\theta^*$ follows (not using the Einstein Summation Convention)
$$
\theta^i\theta^{*}_i=-1. 
$$
Now we proceed to compute $\mfe_i$. (Without the definition of coordinate curves)
$\mfe_i$ shall satisfy
$$
\lra{\mfe_i,\mfe_j}=g_{i,j}(\theta)=\p_i\p_j\psi(\theta)=\delta_{i,j}\pp{\theta^i}^{-2}.
$$
In general,  we can take $\mfe_i =\pp{ \theta^i}^{-1}W_i$, where $W=[W_1,\dots,W_n]$ is orthogonal. 
Then, we work on the dual affine coordinate. 
$$
\lra{\mfe^{*i},\mfe^{*j}}=g^{*i,j}(\theta^*)=\delta_{i,j}\pp{\theta^*_i}^{-2}.
$$
Similarly, we can take $\mfe^{*i} =\pp{ \theta^*_i}^{-1}\bar W_i$, where $\bar W=[\bar W_1,\dots,\bar W_n]$  is orthogonal. It is important to note that the following equation can \textbf{not} hold under the assumption that $\lra{\cdot,\cdot}$ is the standard Euclidean inner product.
$$
d\theta =\mfG^* d\theta^*.
$$
This is because it is assumed that
$$
ds = \lra{d\theta,d\theta} = \lra{d\theta^*,d\theta^*}.
$$
Hence, we assume that $d\theta = d\theta^*$. We note the following fact:
$$d\theta^i=g^{*i,j}d\theta^*_j.$$
Combing with  $d\theta=\mfe_id \theta^i$ and $d\theta^*=\mfe^{*i}d \theta_i^*$, this yields
$$
d\theta =\mfe_id\theta^i=\mfe_ig^{*i,j}d\theta_j^*,%\pp{ \theta^i}^{-1}W_id\theta^i=\pp{ \theta^i}^{-1}W_ig^{*i,j}d\theta^*_j=(\theta^i)^{-3}W_id\theta_i^*
$$
$$
d\theta = d\theta^*=\mfe^{j*}d\theta_j^*,
$$
which implies
$$
\mfe_ig^{*j,i}=\mfe_ig^{*i,j}=\mfe^{j^*}.
$$
(Because $G^*$ is symmetric.) Hence, we have 
$$
\lra{\mfe_i,\mfe^{*j}}=\lra{\mfe_i,g^{*j,k}\mfe_k}=g^{*j,k}g_{i,k}=g_{i,k}g^{*k,j}=\delta_{i,j}
$$
In our example, this tells that
$$
(\theta_j^*)^{-2}(\theta^j)^{-1}W_j=(\theta_j^*)^{-1}\bar W_j,
$$
which tells $W_j=-\bar W_j$. Hence, we can also compute that
$$
\lra{\mfe_i,\mfe^{*j}}=\frac{1}{\theta^i\theta_j^*}\lra{W_i,\bar W_j}=\frac{-1}{\theta^i\theta_j^*}\lra{W_i,W_j}=\delta_{i,j}.
$$
\textbf{Summary} The previous derivation in this example fully based on the assumption that $\lra{\cdot,\cdot}$ is the standard Euclidean inner product. If this is not the case, then it is quite wield to define the inner product because $\lra{\cdot,\cdot}$ takes both the primal and dual affine coordinates.
\end{example}

\section{Exponential families and mixture families of probability distribution}
\subsection{Exponential family}
We first collect several important definitions for exponential families. Consider the parametric probability density
$$
p(x,\theta) = \exp\{\theta^ih_i(x)-\psi(\theta)+k(x)\}.
$$
I would like to use $h_i$ instead of $x_i$ to denote $h_i(x)$. We can define a measure in sample space $\{h\}$, 
$$
d\mu(h) = \exp(k(x))dx.
$$
And we can write
$$
p(x,\theta)dx = p(h,\theta)d\mu(h)= \exp(\theta^Th-\psi(\theta))d\mu(h).
$$
We note that $\psi$ is a normalizing term which satisfies
$$
\psi(\theta) = \log\int \exp(\theta^Th)d\mu(h)
$$
The dual parameter is given by the Legendre transformation $\eta=\nabla \psi(\theta)$, which is the expectation of $h$ (or $x$):
$$
\eta = \mbE[h] = \int h p(h,\theta)d\mu(h)
$$
The Legendre dual $\phi(\eta)$ of $\psi(\theta)$ follows
$$
\phi(\eta) = \theta^T\eta-\psi(\theta) = \int p(h,\theta)\log p(h,\theta) d\mu(h).
$$
This is because $\log p(h,\theta)=\theta^Th-\psi(\theta)$. The metric follows
$$
g_{i,j}(\theta) = \p_i\p_j\psi(\theta) = \mbE_h[\p_i\log p(h,\theta)\p_j\log p(h,\theta)].
$$
\subsection{Mixture family}
The density takes the form
$$
p(x,\eta)=\sum_{i=0}^n\eta_iq_i(x), \quad \sum_{i=0}^nq_i=1,\quad q_i>0
$$

\printbibliography
\end{document}